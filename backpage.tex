\phantomsection
\pdfbookmark[0]{Back page}{backpage}
\renewcommand*{\hrulefill}[1][-0.2mm]{\leavevmode \leaders \hrule height #1 \hfill \kern 0pt}

% /!\ Careful with this, it's hand tuned...
\changepage{7em}{9em}{-7em}{-4em}{-10em}{-5em}{}{}{}
%---------------------------------------

%---------------------------------------
% English
%---------------------------------------

% \section*{TITLE}
\noindent{\Large\bf Leveraging Hardware Construction Languages for Flexible Design Space Exploration on FPGA}

\vspace*{\fill}
% \section*{SUMMARY}
{\normalsize
\noindent{\bf Abstract} --- FPGA based accelerators are imposing themselves as energy efficient alternatives to general purpose CPUs.
However, the hardware development methodologies are still way behind their software counterparts, and initiatives are to be taken in order to increase the productivity of hardware developers.
In this thesis, we explore the possibilities that the emerging Hardware Construction Languages paradigm can bring to the hardware world, notably by leveraging high level features such as functional programming or object oriented development.
We start with a comprehensive analysis of the estimation metrics and methodologies in the context of FPGA development, and then put a particular focus on how such paradigm can be used for design space exploration, introducing two complementary methodologies --- {\it meta design} and {\it meta exploration} --- for such usage.
A software demonstrator, QECE, has been developed and used to demonstrate the usability of those methodologies in various use cases, thanks to a custom benchmark made  of representative applicative kernels.

This thesis is an initiative to enhance hardware developers expressivity, providing them with powerful features such as functional programming and object-oriented development.
}
\vspace*{\fill}%

% \section*{KEYWORDS}
\noindent{\normalsize\it {\bf Keywords:} FPGA, hardware accelerators, Chisel, design space exploration}
~\\

%---------------------------------------
% \hrulefill[1pt]
\vspace{-0.25cm}
\hrule height 1pt
\vspace{0.3cm}
%---------------------------------------

%---------------------------------------
% Français
%---------------------------------------

% \section*{TITRE}
\noindent{\Large\bf Utilisation de langages de construction mat\'erielle pour une exploration flexible des espaces de conception sur FPGA}

\vspace*{\fill}
% \section*{R\'ESUM\'E}
{\normalsize
\noindent{\bf R\'esum\'e} --- Les accélérateurs matériels à base de FPGA s'imposent actuellement comme une alternative à haute efficacité énergétique aux processeurs généralistes classiques.
Cependant, les méthodologies de développement matériel souffrent d'un grand retard par rapport à leurs pendants logiciels, et des initiatives sont nécessaires afin d'accroire la productivité des concepteurs matériels.
Dans cette thèse, nous explorons les possibilités que les nouveaux langages de construction matérielle ouvrent pour le monde de la conception numérique, notamment en permettant l'usage de fonctionnalités de haut niveau telles que la programmation fonctionnelle ou le développement orienté objet.
Nous proposons tout d'abord une analyse de différentes métriques et méthodologies d'estimation pour le développement sur FPGA, et nous intéressons ensuite plus particulièrement à ce que ces nouveaux langages peuvent apporter au domaine de l'exploration d'espace de conception, en introduisant deux méthodologies complémentaires: la {\it méta conception} et la {\it méta exploration}.
Un logiciel démonstrateur, nommé QECE, est développé et utilisé afin de démontrer l'utilisabilité de ces méthodologies sur différents cas d'utilisation, gr\^ace à un ensemble de noyaux applicatifs que nous avons développés.

Cette thèse est une initiative pour améliorer l'expressivité des développeurs matériels, en leur fournissant des fonctionnalités à fort potentiel telles que la programmation fonctionnelle ou le développement orienté objet.
}
\vspace*{\fill}%

% \section*{MOTS-CL\'ES}
\noindent{\normalsize\it {\bf Mots-cl\'es:} FPGA, acc\'el\'erateurs mat\'eriel, Chisel, exploration d'espace de conception}
~\\

%---------------------------------------
% \hrulefill[3pt]
%---------------------------------------

% \section*{ISBN: }
\centering
Thesis prepared at TIMA laboratory

46 Avenue F\'elix Viallet, 38031, GRENOBLE Cedex, France

% {\bf ISBN:}
% \vspace{-6em}
\vspace{-4em}

%---------------------------------------
